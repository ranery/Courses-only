\documentclass[UTF-8, a4paper, 10pt]{article}

\usepackage{xeCJK}
\usepackage{graphicx}
\graphicspath{{figure/}}
\usepackage[unicode]{hyperref}
\hypersetup{colorlinks=true,linkcolor=black}
\usepackage{cite}
\usepackage{indentfirst}
\usepackage{amsmath}
\numberwithin{equation}{section}
\usepackage{listings}
\usepackage{xcolor}
\usepackage{fontspec}
\usepackage{courier}
\usepackage{tikz-timing}

\lstset{
    basicstyle=\scriptsize\ttfamily,
    numbers=left,                                        % 在左侧显示行号
    keywordstyle=\color[RGB]{40,40,255},                 % 设定关键字颜色
    frame=trbl,
    numberstyle=\scriptsize\color{darkgray},           % 设定行号格式
    commentstyle=\it\color[RGB]{0,96,96},                % 设置代码注释的格式
    stringstyle=\ttfamily\slshape\color[RGB]{128,0,0},   % 设置字符串格式
    showstringspaces=false,                              % 不显示字符串中的空格
    language=python,                                     % 设置语言
}


\linespread{1.0}
\setlength{\parskip}{0.5\baselineskip}

\makeatletter
\let\@afterindentfalse\@afterindenttrue
\@afterindenttrue
\makeatother
\setlength{\parindent}{2em}

\addtolength{\topmargin}{-70pt}
\setlength{\oddsidemargin}{0.63cm}
\setlength{\evensidemargin}{\oddsidemargin}
\setlength{\textwidth}{15.66cm}
\setlength{\textheight}{25.00cm}

\newcommand{\chuhao}{\fontsize{42pt}{\baselineskip}\selectfont}
\newcommand{\xiaochuhao}{\fontsize{36pt}{\baselineskip}\selectfont}
\newcommand{\yihao}{\fontsize{28pt}{\baselineskip}\selectfont}
\newcommand{\erhao}{\fontsize{21pt}{\baselineskip}\selectfont}
\newcommand{\xiaoerhao}{\fontsize{18pt}{\baselineskip}\selectfont}
\newcommand{\sanhao}{\fontsize{15.75pt}{\baselineskip}\selectfont}
\newcommand{\sihao}{\fontsize{14pt}{\baselineskip}\selectfont}
\newcommand{\xiaosihao}{\fontsize{12pt}{\baselineskip}\selectfont}
\newcommand{\wuhao}{\fontsize{10.5pt}{\baselineskip}\selectfont}
\newcommand{\xiaowuhao}{\fontsize{9pt}{\baselineskip}\selectfont}
\newcommand{\liuhao}{\fontsize{7.875pt}{\baselineskip}\selectfont}
\newcommand{\qihao}{\fontsize{5.25pt}{\baselineskip}\selectfont}

\begin{document}

\begin{titlepage}
    \begin{center}
    \phantom{Start!}
	\vspace{2cm}
	\includegraphics[width=350pt]{HUST.pdf} \\
    \vspace{1cm}
     \center{
       	  \textbf{\yihao 实\quad 验\quad 报\quad 告}\\
       	  \vspace{0.5cm}
          \textbf{\sanhao (2017 / 2018学年\quad 第2学期)}
      }
      \vspace{2.5cm}
      \begin{table}[!hbp]
      \centering
      \renewcommand\arraystretch{1.5}
     	\begin{tabular}{|c|c|}
     		\hline
     		课程名称 & 机器学习导论 \\
     		\hline
     		实验名称 & ~~~~~~Logistic Regression Classifier~~~~~~ \\
     		\hline
     		实验时间&\multicolumn{1}{c|}{2018年5月29日}\\
     		\hline
     		指导教师 & 王邦 \\
     		\hline
     		\end{tabular}     		
       \end{table}
       \vspace{2cm}
      \begin{table}[htbp]
      \centering
      \renewcommand\arraystretch{1.5}
     	\begin{tabular}{|c|c|c|c|}
     		\hline
            \qquad ~~姓名~~~~~  & \qquad ~~游浩然~~~~~  & \qquad 学号~~~~~ & \qquad U201515429~~~~~ \\
     		\hline
     		\end{tabular}
       \end{table}
       \date{2018年5月29日}
     \end{center}
\end{titlepage}

\section{问题重述}
\begin{itemize}
  \item 钞票数据集(Banknote Dataset)涉及根据给定钞票的4个度量的特征。据此预测是真钞还是假钞。
\end{itemize}

\section{Python Code for Logistic regression}
\subsection{logistic}
\begin{lstlisting}[language=python]
# -*- coding: utf-8 -*-
"""
@author : Haoran You

"""
import numpy as np
import matplotlib.pyplot as plt
# from bigfloat import exp
import random
import csv
import os

class logistic():
    def name(self):
        return 'Logistic Model'

    def sigmoid(self, x):
        return 1.0 / (1 + np.exp(x))

    def stoGradAscent(self, data, label, numIter):
        m, n = np.shape(data)
        weight = [float(1.0) for i in range(n)]
        for j in range(numIter):
            dataIndex = list(range(m))
            for i in range(m):
                # alpha = 4 / (1.0+j+i) + 0.0001
                alpha = 0.00001
                randIndex = int(random.uniform(0, len(dataIndex)))
                h = self.sigmoid(np.sum(np.dot(data[randIndex], weight)))
                error = label[randIndex] - h
                weight += alpha * error * np.array(data[randIndex])
                del(dataIndex[randIndex])
        return weight

    def classifyVector(self, x, weight):
        pred = self.sigmoid(np.sum(np.dot(x, weight)))
        if pred > 0.5: return 1.0
        else: return 0.0

    def train(self, data, label, numIter):
        self.train_data = data
        self.train_label = label
        self.train_weight = self.stoGradAscent(data, label, numIter)

    def val(self, data, label):
        self.val_data = data
        self.val_label = label
        error = 0
        numTest = np.shape(data)[0]
        for i in range(numTest):
            pred = self.classifyVector(data[i], self.train_weight)
            # print(pred, label[i])
            if int(pred) != int(label[i]):
                error += 1
        errorRate = (float(error) / numTest)
        print('Error rate of val data : %f' % errorRate)

    def test(self, data):
        if os.path.exists('results.csv'):
            os.remove('results.csv')
        f = open('results.csv', 'a', newline='')
        csv_write = csv.writer(f, dialect='excel')
        i = 0
        for vector in data:
            result = []
            i += 1
            pred = self.classifyVector(vector, self.train_weight)
            result.append(i)
            for item in vector:
                result.append(item)
            result.append(pred)
            csv_write.writerow(result)
\end{lstlisting}
\subsection{data}
\begin{lstlisting}[language=python]
# -*- coding: utf-8 -*-
"""
@author : Haoran You

"""
import random

def parse_file(filename, has_cls=True):
    f = open(filename, 'r', encoding='gbk')
    data, label = [], []
    for line in f.readlines():
        if has_cls == True:
            data_str = line.strip().split(',')[:-1]
            data_list = []
            for data_item in data_str:
                data_list.append(float(data_item))
            data.append(data_list)
            label.append(float(line.strip().split(',')[-1]))
        else:
            data_str = line.strip().split(',')
            data_list = []
            for data_item in data_str:
                data_list.append(float(data_item))
            data.append(data_list)
    return data, label

def divide(data, label):
    num_train = int(0.8*len(data))
    train_data, train_label = [], []
    val_data, val_label = [], []
    index = random.sample(range(len(data)), num_train)
    for i in range(0, len(data)):
        if i in index:
            train_data.append(data[i])
            train_label.append(label[i])
        else:
            val_data.append(data[i])
            val_label.append(label[i])
    return train_data, train_label, val_data, val_label

def dataset():
    data, label = parse_file('train.txt')
    train_data, train_label, val_data, val_label = divide(data, label)
    test_data, test_label = parse_file('test.txt', has_cls=False)
    print('number of train : ', len(train_data))
    print('number of val   : ', len(val_data))
    print('number of test  : ', len(test_data))
    return train_data, train_label, val_data, val_label, test_data
\end{lstlisting}
\subsection{Main}
\begin{lstlisting}[language=python]
# -*- coding: utf-8 -*-
"""
@author : Haoran You

"""
from data import dataset
from logistic import logistic

# load data
train_data, train_label, val_data, val_label, test_data = dataset()
# train
logistic = logistic()
logistic.train(train_data, train_label, 30)
# val
logistic.val(val_data, val_label)
# test
logistic.test(test_data)
\end{lstlisting}

\end{document} 